\documentclass[titlepage,a4paper]{report}

\usepackage[utf8]{inputenc}
\usepackage[OT1]{fontenc}

\usepackage{illcmolthesis}

% put your own adjustments into headers.tex
% !TeX root = main.tex

% If you need additional packages, add them here.

% math
\usepackage{amsmath}
\usepackage{amsthm}
\usepackage{amssymb}
\usepackage{amsfonts}
\usepackage{mathtools}

% figures and tables
\usepackage{float}
\usepackage{booktabs}

% graphics
\usepackage{graphicx}
\usepackage{tikz}
\usetikzlibrary{backgrounds,positioning,trees,shapes,arrows,patterns,topaths,calc}

% bibliography
\usepackage[
  backend=biber,
  style=alphabetic,
  natbib=true,
  url=true,
  maxcitenames=9,
  maxbibnames=99,
  abbreviate=false,
  eprint=false,
  doi=true,
  backref=true,
]{biblatex}
% Crossref Display Guidelines (March 2017), retrieved 2024-12-30, https://doi.org/10.13003/5jchdy
\DeclareFieldFormat{doi}{\url{https://doi.org/#1}}
\DeclareFieldFormat{url}{\url{#1}}
% If you use multiple .bib files, add them here:
\addbibresource{references.bib}

% Load hyperref after other packages
\usepackage[pdfusetitle]{hyperref}
\hypersetup{
  hidelinks,
  linktoc = all,
  breaklinks = true,
}
\usepackage[nameinlink]{cleveref}

% Theorems etc all using the same counter:
\newtheorem{theorem}{Theorem}
\newtheorem{lemma}[theorem]{Lemma}
\newtheorem{definition}[theorem]{Definition}
\newtheorem{corollary}[theorem]{Corollary}
\newtheorem{remark}[theorem]{Remark}
\newtheorem{example}[theorem]{Example}
\newtheorem{fact}[theorem]{Fact}
\newtheorem{conjecture}[theorem]{Conjecture}

% If you want "Theorem N.M" in section N, activate this:
% \numberwithin{theorem}{chapter}

%%% Local Variables:
%%% mode: latex
%%% TeX-master: "main.tex"
%%% End:


% Please do not actually use this for your thesis ;-)
\usepackage{lipsum}

\begin{document}

\title{TITLE OF THE THESIS}
\author{John~Q.~Public}
% \birthdate{April 1st, 1980} % optional
% \birthplace{Alice Springs, Australia} % optional
\defensedate{August 28, 2005}
\supervisor{Dr Jack Smith}
\supervisor{Prof Dr Jane Williams}
\committeemember{Dr Jack Smith}
\committeemember{Prof Dr Jane Williams}
\committeemember{Dr Jill Jones}
\committeemember{Dr Albert Heijn}
\maketitle

\begin{abstract}
  ABSTRACT OF THE THESIS

  \lipsum[10-11]
\end{abstract}

% If you want to, add acknowledgements *after* your defense.
% Do not add acknowledgements when you upload to DataNose.
% \begin{acknowledgements}
% \end{acknowledgements}

\clearpage

\tableofcontents

\chapter{Introduction}

\lipsum[1]

\section{Background}\label{sec:background}

\lipsum[2]

\section{Literature}

We use standard results from~\cite{BRV2001:Modal}.
Also relevant for our work is~\cite{BB1999:IPGames} where it was proven that Logic is great.

\lipsum[3-10]

\section{Criticism}

\lipsum[11-12]

\chapter{My Logic}

\lipsum[15]

\section{Syntax}

\begin{definition}\label{def:lang}
We defined the \emph{language} $\mathcal{L}$ as follows:
  \[
    \phi ::= \top \mid p \mid \phi \land \phi
  \]
\end{definition}

\lipsum[21]

\section{Semantics}

\section{Axioms}

\section{Soundness}

\section{Completeness}

\chapter{Examples}

\section{Figures}

We illustrate the protagonist of this thesis in \Cref{fig:drawing}.

\begin{figure}[H]
  \centering
  \begin{tikzpicture}
    \draw (0,0) circle (0.5);
    \draw (0,-0.5) -- (0,-2);
    \draw (-1,-1) -- (1,-1);
    \draw (0,-2) -- (-0.5,-3.5);
    \draw (0,-2) -- (0.5,-3.5);
  \end{tikzpicture}
  \caption{The protagonist.}\label{fig:drawing}
\end{figure}

\section{Tables}

In \Cref{tab:cities} we compare some cities.

\begin{table}[H]
  \centering
  \begin{tabular}{l|rr}
    \toprule
    City & Population & area (km$^2$) \\
    \midrule
    Amsterdam & 851573 & 219 \\
    Groningen & 200952 & 83  \\
    Utrecht   & 321989 & 100 \\
    \bottomrule
  \end{tabular}
  \caption{An Overview of Cities}\label{tab:cities}.
\end{table}

\chapter{Conclusion}

As we discussed in \Cref{sec:background}, Logic is great.

\lipsum[40-42]

\clearpage
\printbibliography[heading=bibintoc,title=Bibliography]

\end{document}
